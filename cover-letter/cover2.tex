
\documentclass[letterpaper,10pt]{article}

\usepackage{hyperref}



\begin{document}

High-throughput sequencing of our T cell receptor (TCR) repertoires is revolutionizing our understanding of how we respond to disease and vaccination, and holds great promise for cancer immunotherapy.

Probabilistic models are a key tool for deriving understanding from these sequences.
For example, we recently used a probabilistic model of TCR generation to show that HLA-associated TCRs have lower generation probability than non-HLA associated TCRs, implicating functional selection (DeWitt et.\ al, \emph{eLife} 2018).
Pogorelyy et.\ al (\emph{eLife} 2018) used a similar probabilistic model to identify TCRs responsive to an immune perturbation such as viral infection.
Elhanati et.\ al (\emph{PNAS} 2014) used a related probabilistic model to infer the selective constraints of thymic selection.

The probabilistic models used thus far are classical, formalizing a mental model of VDJ recombination.
As such, they are expressed in terms of conditional probabilities of germline genes, trimming amounts, and nucleotide additions.
Such probabilistic models have been used for over a decade (Ga\"eta et.\ al, \emph{Bioinformatics} 2007), and the current line of work for TCRs follows from the beautiful work of the Mora-Walczak-Callan group (Murugan et.\ al, \emph{PNAS} 2012).
This work has been very successful, leading to many applications and extensions.
However, it is not the only way to proceed in 2019.
We wondered: \emph{can probabilistic models built on deep neural networks learn the rules of VDJ recombination and capture TCR diversity in a generalizable way?}

In our manuscript ``Deep generative models for T cell receptor protein sequences'' we clearly answer this question in the affirmative.
We show that appropriately trained simple deep learning models:
\vspace{-4pt}
\begin{itemize}
  \item can perform accurate cohort frequency estimation for TCRs
  \item can learn the rules of VDJ recombination
  \item generalize well to unseen sequences
  \item are able to distinguish between real sequences and sequences generated according to the Mora-Walczak-Callan model of recombination with a simple selection step, even though this was not a training objective
  \item have similar summary characteristics to real sequences.
\end{itemize}

Turning now to the questions posed on the eLife website:
\textit{How will your work make others in the field think differently and move the field forward?}
All previous probabilistic models, which have played an essential role in understanding T cell receptor diversity, have been classical models developed from a mathematical model of VDJ recombination.
Here we introduce a paradigm shift by showing that very simple deep models, which have no built-in concepts from immunology, perform as well or better than models which incorporate considerable immunological knowledge.
We expect that deep models, with their greater accuracy, will replace existing models for many applications.

\vspace{4pt} \textit{How does your work relate to the current literature on the topic?}
We compare our model rigorously to the Mora-Walczak-Callan model, which is clearly the dominant probabilistic model for TCRs, showing superiority of the deep model to this previous model in many respects.

There is one preprint we know of that does deep learning on TCRs:

\url{qwety}

Sidhom, J.-W., Benjamin Larman, H., Pardoll, D. M., \& Baras, A. S. (2018). DeepTCR: a deep learning framework for revealing structural concepts within TCR Repertoire. \emph{bioRxiv}. 

Sidhom, J.-W., Benjamin Larman, H., Pardoll, D. M., \& Baras, A. S. (2018). DeepTCR: a deep learning framework for revealing structural concepts within TCR Repertoire. \emph{bioRxiv}. \url{https://doi.org/10.1101/464107}

Although this preprint uses some of the same tools as our manuscript, it has a different goal.
Rather than evaluate the probability of seeing a TCR in a repertoire, these authors work to develop tools that can be used for discriminating TCRs based on function.
Furthermore, we do not think that this work is of sufficiently high quality to merit citation.
It makes unsubstantiated claims such as that their tool is ``allowing for the first time a direct and wholistic [sic] comparison of antigen-specific repertoires that leverages a collection of sequences to identify common structural signatures within an antigen-specific response.''
We honestly cannot find any meaningful and non-trivial structure in the figure they are describing (Figure 2).
In fact, the mixing of the colored bars in their Figure 2a and 2e clearly show that their model is not learning meaningful structure in the latent space.

On the other hand, we have cited works by Sinai et.\ al (\emph{NeurIPS}, 2017) and Riesselman et.\ al (\emph{Nature Methods}, 2018) which are stronger papers concerning generative models for proteins in general.


\vspace{4pt} \textit{Who do you consider to be the most relevant audience for this work?}
Immunologists studying the adaptive immune system will wish to use our models to understand TCR frequency expansion and predict TCR naive frequency.
We also believe that computational researchers in other fields will want to learn from our experience on how to use deep neural networks for protein sequences.

\vspace{4pt} \textit{Have you made clear in the letter what the work has and has not achieved?}
We have summarized the strengths of our model above.
However, it has a clear drawback of being uninterpretable.
Certainly the network weights do not have an easy interpretation, and the basic design of the models means that they offer little mechanistic insight.
In addition, the models are exclusively about amino acids and thus cannot shed light on the VDJ recombination process, which happens in terms of nucleotides.

We also highlight some shortcomings of the model in the manuscript, which show that there are further opportunities to improve our deep generative model.

\vspace{4pt}
Thank you for considering our manuscript for publication in eLife.
\end{document}